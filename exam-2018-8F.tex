\documentclass[a4paper,12pt]{article}
\usepackage[utf8x]{inputenc}
\usepackage[russian]{babel}
\usepackage[pdf]{graphviz}
\usepackage{float}
\usepackage{amsmath}

\begin{document}
Зачёт 8Ф 2018 год.\newline
1. Целочисленные типы данных ($int$, $long long$, $unsigned int$). Операции $+$, $-$, $*$, $/$, $\%$.\newline
2. Длина Московской кольцевой автомобильной дороги - $109$ километров. Байкер Вася стартует с нулевого километра МКАД и едет со скоростью $v$ километров в час. На какой отметке он остановится через $t$ часов?\newline 
3. Типы данных с плавающей точкой ($float$, $double$). Сравнение двух вещественных чисел на равенство. Вывод числа на экран с $n$ знаками после запятой.\newline
4. Даны три действительных числа: $a$, $b$, $c$. Проверьте, выполняется ли равенство $a + b = c$. Если равенство выполняется, выведите $YES$, если не выполняется, выведите $NO$. Числа $a$, $b$, $c$ –  действительные, положительные, не превосходят $10$ и заданы не более, чем с $7$ знаками после точки.\newline
5. Тип данных $char$. Тип данных $string$.\newline
6. Необходимо вывести  строку $yes$, если символ является цифрой, и строку $no$ в противном случае.\newline
7. Если символ является строчной буквой латинского алфавита (то есть буквой от $a$ до $z$), выведите вместо него аналогичную заглавную букву, иначе выведите тот же самый символ.\newline
8. Юлий Цезарь использовал свой способ шифрования текста. Каждая буква заменялась на следующую по алфавиту через $K$ позиций по кругу. Необходимо по заданной шифровке определить исходный текст.\newline
9. Дана строка, Вам требуется преобразовать все идущие подряд пробелы в один.\newline
10. Тип данных $bool$. Констант True и False.\newline
11. Приведение типов.\newline
12. Операторы $==$, $!=$, $<$, $>$, $<=$, $>=$, $\&$, $|$, \textasciicircum, $\&\&$, $||$.\newline
13. Оператор $if$.\newline
14. Даны три натуральных числа $a$, $b$, $c$, записанные в отдельных строках. Определите, существует ли неворожденный треугольник с такими сторонами.\newline
15. Дано три числа, записанный в отдельных строках. Упорядочите их в порядке неубывания. Программа должна считывать три числа $a$, $b$, $c$, затем программа должна менять их значения так, чтобы стали выполнены условия $a \le b \le c$, затем программа выводит тройку $a$, $b$, $c$.\newline
16. Цикл $for$.\newline
17. По данному натуральному $n$ вычислите сумму $1^2+2^2+\dots+n^2$.\newline
18. Выведите все числа на отрезке от $a$ до $b$, являющиеся полными квадратами. Если таких чисел нет, то ничего выводить не нужно.\newline
19. Оператор $while$.\newline
20. Дано натуральное число $N$. Выведите слово $YES$, если число $N$ является точной степенью двойки, или слово $NO$ в противном случае.\newline 
21. Вводится последовательность целых чисел. Ввод завершается, когда будет введено число $0$. Определите среднее арифметическое элементов последовательности, завершающейся числом $0$. Число $0$ в последовательность не входит. Числа, следующие за нулем, считывать не нужно.\newline
22. Массив.\newline
23. Дан массив, состоящий из целых чисел. Напишите программу, которая подсчитывает количество положительных чисел среди элементов массива.\newline
24. Проверьте, является ли двумерный массив симметричным относительно главной диагонали. Главная диагональ - та, которая идёт из левого верхнего угла двумерного массива в правый нижний.\newline
25. Функции.\newline
26. Напишите функцию\newline
bool Xor (bool x, bool y) {}, реализующую функцию $Исключающее ИЛИ$ двух логических переменных $x$ и $y$. Функция $Xor$ должна возвращать $true$, если ровно один из ее аргументов $x$ или $y$, но не оба одновременно равны $true$.\newline
27. Напишите $функцию голосования$\newline
bool Election(bool x, bool y, bool z) {}, возвращающую то значение ($true$ или $false$), которое среди значений ее аргументов $x$, $y$, $z$ встречается чаще.\newline
28. Пространство имен.\newline
29. Рекурсия.\newline
30. Последовательность Фибоначчи определена следующим образом: $\phi_0=1$, $\phi_1=1, \phi_n=\phi_{n-1}+\phi_{n-2}$ при $n>1$. Начало ряда Фибоначчи выглядит следующим образом: $1$, $1$, $2$, $3$, $5$, $8$, $13$, $21$, $34$, $55$, $\dots$ Напишите рекурсивную функцию\newline
int phi(int n), которая по данному натуральному $n$ возвращает $\phi_n$.\newline
31. Даны два числа. Реализуйте рекурсивную функцию\newline
int gcd(int a, int b), находящую их наибольший общий делитель.\newline
\end{document}
