\documentclass[a4paper,12pt]{article}
\usepackage[utf8x]{inputenc}
\usepackage[russian]{babel}
\usepackage[pdf]{graphviz}
\usepackage{float}
\usepackage{amsmath}
\usepackage{listings}
\usepackage{xcolor}

\lstset{
    frame=tb,
    tabsize=4,
    showstringspaces=false,
    numbers=left,
    commentstyle=\color{green},
    keywordstyle=\color{blue},
    stringstyle=\color{red}
}

\newtheorem{definition}{Опредление} % задаём выводимое слово (для определений)

\begin{document}
Оценка времени работы программы. O-нотация.
При выборе алгоритма, решающего поставленную задачу всегда нужно оценить время его работы на входных данных. Рассмотрима следующий код:
\begin{lstlisting}[language=C++]
#include <iostream>

using namespace std;

int main() {
    int x;
    cin >> x;
    cout << x * x << endl;
    return 0;
}
\end{lstlisting}
Условимся считать, что подключение библиотеки и "using namespace std;" происходит мгновенно, а все остальные операции стоят одну еденицу времени
(На самом деле это не верно. Например, умножение двух чисел происходит дольше, чем их сложение. Самыми же доргими операциями явлеются операции
чтении и записи. Но для наших задач такое пренебрежение допустимо). Тогда мы тратим одну еденицу времени на создание переменной (int x;),
одну на чтение (cin >> x;), одну на умножение (x * x), одну на запись (cout), одну на return. Итого время работы первой программы $T1 = 5$ едениц
времени.\newline
Рассмотрим второй пример.
\begin{lstlisting}[language=C++]
#include <iostream>

using namespace std;

int main() {
    int N;
    cin >> N;
    for (int i = 0; i < N; ++i) {
        int x;
        cin >> x;
        cout << x * x << endl;
    }
    return 0;
}
\end{lstlisting}
Одна еденица времени тратиться на создание переменной N, одна на чтение, одна на создание i, одна на присваивание 0 переменной i. Далее, N раз
происходит следующие операции: сравнение i и N, создание перменной x, её считывание, возведение в квадрат, вывод $x^2$, увеличение i на 1. Потом
в И + 1 раз сравниться i и N, после чего программы выйдет из for и сделает "return 0;". Итого, время работы второй программы равно $T2 = 1 + 1 + 1
+ 1 + N(1 + 1 + 1 + 1 + 1 + 1) + 1 + 1 = 6N + 6$.\newline
Рассмотрим третий пример.
\begin{lstlisting}[language=C++]
#include <iostream>

using namespace std;

int main() {
    int N;
    cin >> N;
    for (int i = 0; i < N; ++i) {
        for (int j = 0; j < N; ++j) {
            int x;
            cin >> x;
            cout << x * x << endl;
        }
    }
    return 0;
}
\end{lstlisting}
Одна еденица времени тратиться на создание переменной N, одна на чтение, одна на создание i, одна на присваивание 0 переменной i. Далее, N раз
происходит следующие операции: сравнение i и N, выполнение второго цикла for, увеличение i на 1. Потом в И + 1 раз сравниться i и N, после чего
программы выйдет из for и сделает "return 0;". Из предыдущего примера видно, что второй for делает $6N + 3$ операции. Итого, время работы третьей
программы равно $T3 = 1 + 1 + 1 + 1 + N(1 + (6N + 3) + 1) + 1 + 1 = N(6N + 5) + 6 = 6N^2 + 5N + 6$.\newline
Как видите, посчитать время работы довольно сложно, даже пренебригая разным времени работы различных операций. Поэтому время работы программы
считают "примерно".\newline
Что происходит со временем работы программы при больших N? Например, N = 1000000.\newline
1) $T1 = 5$ - "константа"\newline
2) $T2 = 6N + 6 = 6000006 \approx 6000000 = 6N$ - "константа умножить на $N$"\newline
3) $T3 = 6N^2 + 5N + 6 = 6000005000006 \approx 6000000000000 = 6N^2$ - "константа умножить на $N^2$"\newline
В первом случае $T1$ не изменилось, T2 оказалось примрно равно $6N$, T3 оказалось примрно равно $6N^2$. Т.е. при очень больших числах нам
оказался интересен одночлен с наибольшей степенью, он вносит наибольший вклад в работу программы. На самом деле, нам так же не очень интерсна
константа при этом одночлене, потому что она всегда будет небольшая. Поэтому говорят, что первая программа работает за $O(1)$ (или за константу),
вторая за $O(N)$ (или за линейное время), третья за $O(N^2)$ (или за квадрат). Что же такое $O(g(N))?$\newline
Ниже будет определение $O(g(N))$. Но срауз оговрюсь, что оно будет не полностью полным и строгим. Для строго определения нужно знать начала
математического анализа.\newline
\begin{definition}
Пусть $N$ - вещественное число; $f$, $g$ - функции.
$f(N) = O(g(N))$ (говоят, что $f$ - это $O$ от $g$), если существует такие вещественные положительные числа $N_0$ и $C$, что для любых
вещественных $N' > N_0$, выполняется неравенство $|f(N')| \le C|g(N')|$.
\end{definition}
Давайте проверим, что $5 = O(1)$, $6N + 6 = O(N)$, $6N^2 + 5N + 6 = O(N^2)$:\newline
1) $C = 5$, $N_0 = 1$, тогда $5 <= 5$ для всех $N' > 0$\newline
2) $C = \frac{1}{7}$, $N_0 = 6$, тогда $6N + 6 <= 7N$ для всех $N' > 6$\newline
3) $C = \frac{1}{7}$, $N_0 = 6$, тогда $6N^2 + 5N + 6 <= 7N^2$ для всех $N' > 6$\newline
Как же понять, какой алгоритм подходит для решение задачи? Асимптотика времени работы не должна быть не больше $10^6-10^7$ для самого большого $N$.
Т.е. если $N \le 100$, то $f(N)$ должно быть не больше $O(N^3)$ ($O(N^4)$ уже не пройдёт), $N \le 1000$, то $f(N) \le O(N^2)$, $N \le 10^6$, то
$f(N) \le O(N)$, $N \le 10^{12}$, то $f(N) \le O(\sqrt{N})$, $N \le 10^{18}$, то $f(N) \le O(\sqrt[3]{N})$(в последнем случае чаще всего
предполагается $f(N) \le O(1)$).\newline
Ещё немного примеров: что объеденяет следющие программы?\newline
\begin{lstlisting}[language=C++]
#include <iostream>

using namespace std;

int main() {
    int N;
    cin >> N;
    for (int i = 0; i < N; ++i) {
        int x;
        cin >> x;
        cout << x * x << endl;
    }
    return 0;
}
\end{lstlisting}
\begin{lstlisting}[language=C++]
#include <iostream>

using namespace std;

int main() {
    int N;
    cin >> N;
    for (int i = 0; i < N / 2; ++i) {
        int x;
        cin >> x;
        cout << x * x << endl;
    }
    return 0;
}
\end{lstlisting}
\begin{lstlisting}[language=C++]
#include <iostream>

using namespace std;

int main() {
    int N;
    cin >> N;
    for (int i = 0; i < N; i += 2) {
        int x;
        cin >> x;
        cout << x * x << endl;
    }
    return 0;
}
\end{lstlisting}
\begin{lstlisting}[language=C++]
#include <iostream>

using namespace std;

int main() {
    int N;
    cin >> N;
    for (int i = 0; i < N; ++i) {
        int x;
        cin >> x;
        cout << x * x << endl;
    }
    for (int i = 0; i < N; ++i) {
        int x;
        cin >> x;
        cout << x * x << endl;
    }
    return 0;
}
\end{lstlisting}
Время работы каждой из программ равно $O(N)$.\newline
Чему равно время работы биномиального поиска? Примерно $\log_{2}N$ или $O(log N)$. Почему пропустили 2 в предыдущем выражении? Потому что
$\log_{c}N = \frac{\log_{2}N}{\log_{c}{2}} = constant\log_2{N}$. Поэтому $\log_{c}N = O(\log_2{N})$, поэтому принято 2 опускать.

\end{document}
