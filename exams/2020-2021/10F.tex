\documentclass[a4paper,12pt]{article}
\usepackage[utf8x]{inputenc}
\usepackage[russian]{babel}
\usepackage[pdf]{graphviz}
\usepackage{float}
\usepackage{amsmath}
\usepackage{hyperref}

\begin{document}
Зачёт 10Ф 2021 год.
\begin{enumerate}
\item Решето Эратосфена. \href{https://informatics.msk.ru/mod/statements/view3.php?chapterid=112450}{(контест 42, задача C)}
\item Возведение $x$ в степень $N$ по модулю $P$. \href{https://informatics.msk.ru/mod/statements/view3.php?chapterid=111741}{(контест 43, задача B)}
\item Битовые операции.
\item Напишите программу, которая инвертирует определенный бит в заданном числе (биты при этом нумеруются с 0, начиная с младших). \href{https://informatics.msk.ru/mod/statements/view3.php?chapterid=111741}{(контест 44, задача E)}
\item Напишите программу, которая обнуляет заданное количество последних бит числа. \href{https://informatics.msk.ru/mod/statements/view3.php?chapterid=121}{(контест 44, задача G)}
\item Векторное и скалярное произведения.
\item Расстояние от точки до отрезка. \href{https://informatics.msk.ru/mod/statements/view3.php?chapterid=279}{(контест 45, задача B)}
\item Уравнение прямой. \href{https://informatics.msk.ru/mod/statements/view3.php?chapterid=270}{(контест 46, задача A)}
\item Расстояние от точки до прямой. \href{https://informatics.msk.ru/mod/statements/view3.php?chapterid=270}{(контест 46, задача E)}
\item Особые точки многоугольников.
\item Касательная к окружности. \href{https://informatics.msk.ru/mod/statements/view3.php?chapterid=283}{(контест 47, задача G)}
\item Площадь многоугольника. \href{https://informatics.msk.ru/mod/statements/view3.php?chapterid=446}{(контест 48, задача B)}
\item Лежит ли точка внутри многоугольника. \href{https://informatics.msk.ru/mod/statements/view3.php?chapterid=288}{(контест 48, задача D)}
\item Выпуклая оболочка. \href{https://informatics.msk.ru/mod/statements/view3.php?chapterid=638}{(контест 48, задача E)}
\end{enumerate}
\end{document}
