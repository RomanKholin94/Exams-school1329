\documentclass[a4paper,12pt]{article}
\usepackage[utf8x]{inputenc}
\usepackage[russian]{babel}
\usepackage[pdf]{graphviz}
\usepackage{float}
\usepackage{amsmath}
\usepackage{hyperref}

\begin{document}
Зачёт 8Ф 2019 год.
\begin{enumerate}
\item Ассимптотика времени работы программы. $O$-нотация.
\item Напишите программу, которая определяет, сколько раз встречается заданное число x в данном массиве. \href{https://informatics.msk.ru/mod/statements/view3.php?chapterid=223}{(контест 18, задача A)}
\item Напишите программу, которая находит номер максимального элемента массива. \href{https://informatics.msk.ru/mod/statements/view3.php?chapterid=228}{(контест 18, задача E)}
\item Бинарный поиск.\href{https://informatics.msk.ru/mod/statements/view3.php?chapterid=111728}{(контест 19, задача A)}
\item Сегодня утром жюри решило добавить в вариант олимпиады еще одну, Очень Легкую Задачу. Ответственный секретарь Оргкомитета напечатал ее условие в одном экземпляре, и теперь ему нужно до начала олимпиады успеть сделать еще $N$ копий. В его распоряжении имеются два ксерокса, один из которых копирует лист за $x$ секунд, а другой – за $y$. (Разрешается использовать как один ксерокс, так и оба одновременно. Можно копировать не только с оригинала, но и с копии.) Помогите ему выяснить, какое минимальное время для этого потребуется. $1\le N\le 2*10^8, 1\le x, y \le 10$. \href{https://informatics.msk.ru/mod/statements/view3.php?chapterid=490}{(контест 19, задача F)}
\item Сортировки Выбором.
\item Сортировки Пузырьком.
\item Сортировки Вставками.
\item Во время проведения олимпиады каждый из участников получил свой идентификационный номер – натуральное число. Необходимо отсортировать список участников олимпиады по количеству набранных ими баллов. \href{https://informatics.msk.ru/mod/statements/view3.php?chapterid=1446}{(контест 20, задача G)}
\item Сортировка Слиянием.
\item Быстрая сортировка.
\item Назовем два массива похожими, если они состоят из одних и тех же элементов (без учета кратности). По двум данным массивам выясните, похожие они или нет. Длины массивов $\le 10^5$. \href{https://informatics.msk.ru/mod/statements/view3.php?chapterid=767}{(контест 21, задача C)}
\item Мальчик подошел к платной лестнице. Чтобы наступить на любую ступеньку, нужно заплатить указанную на ней сумму. Мальчик умеет перешагивать на следующую ступеньку, либо перепрыгивать через ступеньку. Требуется узнать, какая наименьшая сумма понадобится мальчику, чтобы добраться до верхней ступеньки. $N$ - количество ступенек, $\le 100$, стоимость каждой ступеньки меньше 100. \href{https://informatics.msk.ru/mod/statements/view3.php?chapterid=915}{(контест 22, задача E)}
\item Кузнечик прыгает по столбикам, расположенным на одной линии на равных расстояниях друг от друга. Столбики имеют порядковые номера от $1$ до $N$. В начале Кузнечик сидит на столбике с номером $1$. Он может прыгнуть вперед на расстояние от $1$ до $K$ столбиков, считая от текущего. Требуется найти количество способов, которыми Кузнечик может добраться до столбика с номером $N$ . Учитывайте, что Кузнечик не может прыгать назад. $1 \le N, K \le 32$. \href{https://informatics.msk.ru/mod/statements/view3.php?chapterid=112603}{(контест 22, задача F)}
\item В каждой клетке прямоугольной таблицы $N\times M$ записано некоторое число. Изначально игрок находится в левой верхней клетке. За один ход ему разрешается перемещаться в соседнюю клетку либо вправо, либо вниз (влево и вверх перемещаться запрещено). При проходе через клетку с игрока берут столько килограммов еды, какое число записано в этой клетке (еду берут также за первую и последнюю клетки его пути). Требуется найти минимальный вес еды в килограммах, отдав которую игрок может попасть в правый нижний угол. \href{https://informatics.msk.ru/mod/statements/view3.php?chapterid=944}{(контест 23, задача B)}
\item Дана прямоугольная доска N × M (N строк и M столбцов). В левом верхнем углу находится шахматный конь, которого необходимо переместить в правый нижний угол доски. При этом конь может ходить 4 способами: вверх на 1 и вправо на 2 клетки, вниз на 1 и вправо на 2 клетки, вниз на 2 и на вправо на 1 клетки, вниз на 2 влево на 1 клетки. Необходимо определить, сколько существует различных маршрутов, ведущих из левого верхнего в правый нижний угол. \href{https://informatics.msk.ru/mod/statements/view3.php?chapterid=2962}{(контест 23, задача E)}
\item Наибольшей возрастающая подпоследовательность. \href{https://informatics.msk.ru/mod/statements/view3.php?chapterid=205}{(контест 24, задача D)}
\item Наибольшая общая подпоследовательность. \href{https://informatics.msk.ru/mod/statements/view3.php?chapterid=204}{(контест 24, задача A)}
\item Количество правильных скобочных последовательностей. \href{https://informatics.msk.ru/mod/statements/view3.php?chapterid=3005}{(контест 24, задача F)}
\item Задача о рюкзаке. \href{https://informatics.msk.ru/mod/statements/view3.php?chapterid=3089}{(контест 25, задача A)}
\item Cтек. \href{https://informatics.msk.ru/mod/statements/view3.php?chapterid=54}{(Условие задачи)}
\item Очередь. \href{https://informatics.msk.ru/mod/statements/view3.php?chapterid=57}{(контест 35, задача A)}
\item Дек. \href{https://informatics.msk.ru/mod/statements/view3.php?chapterid=60}{(контест 37, задача A)}
\item Куча. Пирамидальная сортировка.
\item Теория графов. Матрица смежности. Списки смежности.
\item Обход в глубину(он же DFS, он же Depth-first search).
\item Дан неориентированный невзвешенный граф. Для него вам необходимо найти количество вершин, лежащих в одной компоненте связности с данной вершиной (считая эту вершину). \href{https://informatics.msk.ru/mod/statements/view3.php?chapterid=165}{(контест 29, задача С)}
\item Дан неориентированный невзвешенный граф. Необходимо посчитать количество его компонент связности и вывести их. $N, M \le 10^6$. \href{https://informatics.msk.ru/mod/statements/view3.php?chapterid=111540}{(контест 38, задача A)}
\item Обход в ширину(он же BFS, он же Breadth-first search).
\item В неориентированном графе требуется найти минимальный путь между двумя вершинами. \href{https://informatics.msk.ru/mod/statements/view3.php?chapterid=160}{(контест 30, задача A)}
\item Алгоритм Дейкстры. \href{https://informatics.msk.ru/mod/statements/view3.php?chapterid=5}{(контест 39, задача A)}
\item Алгоритм Флойда - Уоршелла. \href{https://informatics.msk.ru/mod/statements/view3.php?chapterid=171}{(контест 40, задача A)}
\item Алгоритм Алгоритм Беллмана - Форда - Мура. \href{https://informatics.msk.ru/mod/statements/view3.php?chapterid=178}{(контест 41, задача A)}
\item В Летней Компьютерной Школе (ЛКШ) построили аттракцион "Лабиринт знаний". Лабиринт представляет собой N комнат, занумерованных от 1 до N, между некоторыми из которых есть двери. Когда человек проходит через дверь, показатель его знаний изменяется на определенную величину, фиксированную для данной двери. Вход в лабиринт находится в комнате 1, выход – в комнате N. Каждый ученик проходит лабиринт ровно один раз и попадает в ту или иную учебную группу в зависимости от количества набранных знаний (при входе в лабиринт этот показатель равен нулю). Ваша задача показать наилучший результат. $N \le 1000, M \le 10000$. \href{https://informatics.msk.ru/mod/statements/view3.php?chapterid=179}{(контест 41, задача B)}
\item Дан ориентированный граф. Определить, есть ли в нем цикл отрицательного веса, и если да, то вывести его. \href{https://informatics.msk.ru/mod/statements/view3.php?chapterid=180}{(контест 41, задача C)}
\end{enumerate}
\end{document}
