\documentclass[a4paper,12pt]{article}
\usepackage[utf8x]{inputenc}
\usepackage[russian]{babel}
\usepackage[pdf]{graphviz}
\usepackage{float}
\usepackage{amsmath}
\usepackage{hyperref}

\begin{document}
Зачёт 10Ф новички 2020 год.
\begin{enumerate}
\item Целочисленные типы данных ($int$, $long long$, $unsigned int$). Операции $+$, $-$, $*$, $/$, $\%$.
\item Длина Московской кольцевой автомобильной дороги - $109$ километров. Байкер Вася стартует с нулевого километра МКАД и едет со скоростью $v$ километров в час. На какой отметке он остановится через $t$ часов? \href{https://informatics.msk.ru/mod/statements/view3.php?chapterid=2940}{(контест 2, задача D)} 
\item Типы данных с плавающей точкой ($float$, $double$). Сравнение двух вещественных чисел на равенство. Вывод числа на экран с $n$ знаками после запятой.
\item Даны три действительных числа: $a$, $b$, $c$. Проверьте, выполняется ли равенство $a + b = c$. Если равенство выполняется, выведите $YES$, если не выполняется, выведите $NO$. Числа $a$, $b$, $c$ –  действительные, положительные, не превосходят $10$ и заданы не более, чем с $7$ знаками после точки.\href{https://informatics.msk.ru/mod/statements/view3.php?chapterid=74}{(контест 1. Изи кактка, задача F)}
\item Тип данных $char$. Тип данных $string$.
\item Необходимо вывести  строку $yes$, если символ является цифрой, и строку $no$ в противном случае. \href{https://informatics.msk.ru/mod/statements/view3.php?chapterid=102}{(контест 26, задача A)}
\item Если символ является строчной буквой латинского алфавита (то есть буквой от $a$ до $z$), выведите вместо него аналогичную заглавную букву, иначе выведите тот же самый символ. \href{https://informatics.msk.ru/mod/statements/view3.php?chapterid=103}{(контест 26, задача B)}
\item По данной строке определите, является ли она палиндромом (то есть, можно ли прочесть ее наоборот, как, например, слово "топот"). \href{https://informatics.msk.ru/mod/statements/view3.php?chapterid=108}{(контест 26, задача F)}
\item Дана строка, содержащая пробелы. Найдите, сколько в ней слов (слово – это последовательность непробельных символов, слова разделены одним пробелом, первый и последний символ строки – не пробел). \href{https://informatics.msk.ru/mod/statements/view3.php?chapterid=106}{(контест 26, задача E)}
\item Тип данных $bool$. Константы True и False.
\item Приведение типов.
\item Операторы $==$, $!=$, $<$, $>$, $<=$, $>=$, $\&$, $|$, \textasciicircum, $\&\&$, $||$.
\item Оператор $if$.
\item Требуется определить, является ли данный год високосным. (Напомним, что год является високосным, если его номер кратен 4, но не кратен 100, а также если он кратен 400.) \href{https://informatics.msk.ru/mod/statements/view3.php?chapterid=253}{(контест 21, задача E)}
\item Требуется определить, можно ли от шоколадки размером n × m долек отломить k долек, если разрешается сделать один разлом по прямой между дольками (то есть разломить шоколадку на два прямоугольника). \href{https://informatics.msk.ru/mod/statements/view3.php?chapterid=258}{(контест 21, задача G)}
\item Оператор $for$.
\item По данному натуральному $n$ вычислите сумму $1^2+2^2+\dots+n^2$. \href{https://informatics.msk.ru/mod/statements/view3.php?chapterid=315}{(контест 22, задача С)}
\item Выведите все числа на отрезке от $a$ до $b$, являющиеся полными квадратами. Если таких чисел нет, то ничего выводить не нужно. \href{https://informatics.msk.ru/mod/statements/view3.php?chapterid=335}{(контест 22, задача G)}
\item Оператор $while$.
\item Дано натуральное число N. Напишите функцию, вычисляющую сумму цифр числа N. \href{https://informatics.msk.ru/mod/statements/view3.php?chapterid=114}{(контест 23, задача E)}
\item Дано натуральное число N. Выведите его представление в двоичном виде в обратном порядке. \href{https://informatics.msk.ru/mod/statements/view3.php?chapterid=117}{(контест 23, задача F)}
\item Массив.
\item Дан массив, состоящий из целых чисел. Напишите программу, которая в данном массиве определит количество элементов, у которых два соседних и, при этом, оба соседних элемента меньше данного. Дан массив, состоящий из целых чисел. Напишите программу, которая в данном массиве определит количество элементов, у которых два соседних и, при этом, оба соседних элемента меньше данного. \href{https://informatics.msk.ru/mod/statements/view3.php?chapterid=68}{(контест 24, задача A)}
\item Проверьте, является ли двумерный массив симметричным относительно главной диагонали. Главная диагональ - та, которая идёт из левого верхнего угла двумерного массива в правый нижний. \href{https://informatics.msk.ru/mod/statements/view3.php?chapterid=355}{(контест 25, задача A)}
\item Функции.
\item Напишите функцию Xor(bool x, bool y) {}, реализующую функцию $Исключающее ИЛИ$ двух логических переменных $x$ и $y$. Функция $Xor$ должна возвращать $true$, если ровно один из ее аргументов $x$ или $y$, но не оба одновременно равны $true$. \href{https://informatics.msk.ru/mod/statements/view3.php?chapterid=308}{(контест 27, задача C)}
\item Напишите $функцию голосования$ bool Election(bool x, bool y, bool z) {}, возвращающую то значение ($true$ или $false$), которое среди значений ее аргументов $x$, $y$, $z$ встречается чаще. \href{https://informatics.msk.ru/mod/statements/view3.php?chapterid=309}{(контест 27, задача D)}
\item Пространство имен.
\item Рекурсия.
\item Последовательность Фибоначчи определена следующим образом: $\phi_0=1$, $\phi_1=1, \phi_n=\phi_{n-1}+\phi_{n-2}$ при $n>1$. Начало ряда Фибоначчи выглядит следующим образом: $1$, $1$, $2$, $3$, $5$, $8$, $13$, $21$, $34$, $55$, $\dots$ Напишите рекурсивную функцию int phi(int n), которая по данному натуральному $n$ возвращает $\phi_n$. \href{https://informatics.msk.ru/mod/statements/view3.php?chapterid=312}{(контест 28, задача A)}
\item Даны два числа. Реализуйте рекурсивную функцию int gcd(int a, int b), находящую их наибольший общий делитель. \href{https://informatics.msk.ru/mod/statements/view3.php?chapterid=199}{(контест 28, задача B)}
\item Ассимптотика времени работы программы. $O$-нотация.
\end{enumerate}
\end{document}
