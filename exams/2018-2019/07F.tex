\documentclass[a4paper,12pt]{article}
\usepackage[utf8x]{inputenc}
\usepackage[russian]{babel}
\usepackage[pdf]{graphviz}
\usepackage{float}
\usepackage{amsmath}

\begin{document}
Зачёт 7Ф 2019 год.\newline
1. Целочисленные типы данных ($int$, $long long$, $unsigned int$). Операции $+$, $-$, $*$, $/$, $\%$.\newline
2. Ввод-вывод данных.\newline
3. Длина Московской кольцевой автомобильной дороги —109 километров. Байкер Вася стартует с нулевого километра МКАД и едет со скоростью v километров в час. На какой отметке он остановится через t часов? (контест 1, задача I)\newline
4. Улитка ползёт по вертикальному шесту высотой h метров, поднимаясь за день на a метров, а за ночь спускаясь на b метров. На какой день улитка доползёт до вершины шеста? Гарантируется, что a>b. (контест 2, задача H)\newline
5. Условный оператор. Операторы $==$, $!=$, $<$, $>$, $<=$, $>=$, $\&$, $|$, \textasciicircum, $\&\&$, $||$.\newline
6. Даны три натуральных числа a, b, c, записанные в отдельных строках. Определите, существует ли неворожденный треугольник с такими сторонами.(контест 4, задача D)\newline
7. На сковородку одновременно можно положить k котлет. Каждую котлету нужно с каждой стороны обжаривать m минут непрерывно. За какое наименьшее время удастся поджарить с обеих сторон n котлет? Все числа не превосходят 32000. (контест 5, задача G)\newline
8. Операторы for и while.\newline
9. По данному натуральному n вычислите сумму $1^2+2^2+\dots+n^2$. (контест 6, задача С)\newline
10. Найдите все целые решения уравнения $ax^3 + bx^2 + cx + d = 0$ на отрезке $[0,1000]$ и выведите их в порядке возрастания.  Если на данном отрезке нет ни одного решения, то ничего выводить не нужно. (контест 7, задача E)\newline
11. Дано натуральное число $N$. Напишите программу, вычисляющую сумму цифр числа $N$. (контест 8, задача G)\newline
12. Дано натуральное число $N$. Выведите его представление в двоичном виде в обратном порядке. (контест 8, задача H)\newline
13. Типы данных с плавающей точкой (f loat, double). Сравнение двух вещественных чисел на равенство. Вывод числа на экран с $N$ знаками после запятой.\newline
14. Даны три действительных числа, заданы не более, чем с 7 знаками после точки: $a$, $b$, $c$. Проверьте, выполняется ли равенство $a + b = c$. Если равенство выполняется, выведите $YES$, если не выполняется, выведите $NO$. (контест 8, задача A)\newline
15. Массивы.\newline
16. Дан массив, состоящий из целых чисел. Известно, что числа упорядочены по неубыванию (то есть каждый следующий элемент не меньше предыдущего). Напишите программу, которая определит количество различных чисел в этом массиве. (контест 9, задача D)\newline
17. Дана последовательность натуральных чисел $1$, $2$, $3$, $\dots$, $N$ ($1 \le N \le 1000$). Необходимо сначала расположить в обратном порядке часть этой последовательности от элемента с номером $A$ до элемента с номером $B$, а затем от $C$ до $D$ ($A < B$; $C < D$; $1 \le A$, $B$, $C$, $D \le N$). (контест 9, задача G)\newline
18. Проверьте, является ли двумерный массив симметричным относительно главной диагонали. Главная диагональ — та, которая идёт из левого верхнего угла двумерного массива в правый нижний. (контест 10, задача B)\newline
19. В метании молота состязается n спортcменов. Каждый из них сделал $m$ бросков. Победителем считается тот спортсмен, у которого сумма результатов по всем броскам максимальна. Если перенумеровать спортсменов числами от $0$ до $n-1$, а попытки каждого из них – от $0$ до $m-1$, то на вход программа получает массив $A[n][m]$, состоящий из неотрицательных целых чисел. Программа должна определить максимальную сумму чисел в одной строке и вывести на экран эту сумму и номер строки, для которой достигается эта сумма. (контест 10, задача B)\newline
20. Тип данных char и string.\newline
21. Напишите программу, определяющую, является ли данный символ цифрой или нет. (контест 11, задача A)\newline
22. Напишите программу, которая переводит данный символ в верхний регистр. (контест 11, задача B)\newline
23. По данной строке определите, является ли она палиндромом (то есть, можно ли прочесть ее наоборот, как, например, слово "топот"). (контест 11, задача G)\newline
24. Дана строка, Вам требуется преобразовать все идущие подряд пробелы в один. (контест 12, задача E)\newline
25. Функции. Врямя жизни переменных(локальные, глобальные переменные). Рекурсия.\newline
26. Напишите функцию int min (int a, int b, int c, int d), находящую наименьшее из четырех данных чисел. (контест 13, задача A)\newline
27. Напишите рекурсивную функцию int phi(int n), которая по данному натуральному $n$ возвращает $n$-е число Фибоначчи. (контест 14, задача A)\newline
28. Даны два числа. Реализуйте рекурсивную функцию int gcd(int a, int b), находящую их наибольший общий делитель. (контест 14, задача B)\newline
29. Даны два числа. Реализуйте рекурсивную функцию int binomial\_coefficient(int n, int k), находящую число сочетаний из $n$ по $k$. (контест 14, задача C)
\end{document}
