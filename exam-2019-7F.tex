\documentclass[a4paper,12pt]{article}
\usepackage[utf8x]{inputenc}
\usepackage[russian]{babel}
\usepackage[pdf]{graphviz}
\usepackage{float}
\usepackage{amsmath}

\begin{document}
Зачёт 7Ф 2019 год.\newline
1. Целочисленные типы данных ($int$, $long long$, $unsigned int$). Операции $+$, $-$, $*$, $/$, $\%$.\newline
2. Ввод-вывод данных.\newline
3. Длина Московской кольцевой автомобильной дороги —109 километров. Байкер Вася стартует с нулевого километра МКАД и едет со скоростью v километров в час. На какой отметке он остановится через t часов? (контест 1, задача I)\newline
4. Улитка ползёт по вертикальному шесту высотой h метров, поднимаясь за день на a метров, а за ночь спускаясь на b метров. На какой день улитка доползёт до вершины шеста? Гарантируется, что a>b. (контест 2, задача H)\newline
5. Условный оператор. Операторы $==$, $!=$, $<$, $>$, $<=$, $>=$, $\&$, $|$, \textasciicircum, $\&\&$, $||$.\newline
6. Даны три натуральных числа a, b, c, записанные в отдельных строках. Определите, существует ли неворожденный треугольник с такими сторонами.(контест 4, задача D)\newline
7. На сковородку одновременно можно положить k котлет. Каждую котлету нужно с каждой стороны обжаривать m минут непрерывно. За какое наименьшее время удастся поджарить с обеих сторон n котлет? Все числа не превосходят 32000. (контест 5, задача G)\newline
8. Операторы for и while.\newline
9. По данному натуральному n вычислите сумму $1^2+2^2+\dots+n^2$. (контест 6, задача С)\newline
10. Найдите все целые решения уравнения $ax^3 + bx^2 + cx + d = 0$ на отрезке $[0,1000]$ и выведите их в порядке возрастания.  Если на данном отрезке нет ни одного решения, то ничего выводить не нужно. (контест 7, задача E)\newline
11. Дано натуральное число $N$. Напишите программу, вычисляющую сумму цифр числа $N$. (контест 8, задача G)\newline
12. Дано натуральное число $N$. Выведите его представление в двоичном виде в обратном порядке. (контест 8, задача H)\newline
13. Типы данных с плавающей точкой (f loat, double). Сравнение двух вещественных чисел на равенство. Вывод числа на экран с $N$ знаками после запятой.\newline
\end{document}
