\documentclass[a4paper,12pt]{article}
\usepackage[utf8x]{inputenc}
\usepackage[russian]{babel}
\usepackage[pdf]{graphviz}
\usepackage{float}
\usepackage{amsmath}

\begin{document}
Зачёт 8Ф 2019 год.\newline
1. Ассимптотика времени работы программы. $O$-нотация.
2. Напишите программу, которая определяет, сколько раз встречается заданное число x в данном массиве.(контест 18, задача A)\newline
3. Напишите программу, которая находит номер максимального элемента массива.(контест 18, задача E)\newline
4. Бинарный поиск.(контест 19, задача A)\newline
5. Сегодня утром жюри решило добавить в вариант олимпиады еще одну, Очень Легкую Задачу. Ответственный секретарь Оргкомитета напечатал ее условие в одном экземпляре, и теперь ему нужно до начала олимпиады успеть сделать еще $N$ копий. В его распоряжении имеются два ксерокса, один из которых копирует лист за $x$ секунд, а другой – за $y$. (Разрешается использовать как один ксерокс, так и оба одновременно. Можно копировать не только с оригинала, но и с копии.) Помогите ему выяснить, какое минимальное время для этого потребуется. $1\le N\le 2*10^8, 1\le x, y \le 10$. (контест 19, задача F)\newline
6. Сортировки Выбором.\newline
7. Сортировки Пузырьком.\newline
8. Сортировки Вставками.\newline
9. Во время проведения олимпиады каждый из участников получил свой идентификационный номер – натуральное число. Необходимо отсортировать список участников олимпиады по количеству набранных ими баллов. (контест 20, задача G)\newline
10. Сортировка Слиянием.\newline
11. Быстрая сортировка.\newline
12. Назовем два массива похожими, если они состоят из одних и тех же элементов (без учета кратности). По двум данным массивам выясните, похожие они или нет. Длины массивов $\le 10^5$. (контест 21, задача C)\newline
13. Мальчик подошел к платной лестнице. Чтобы наступить на любую ступеньку, нужно заплатить указанную на ней сумму. Мальчик умеет перешагивать на следующую ступеньку, либо перепрыгивать через ступеньку. Требуется узнать, какая наименьшая сумма понадобится мальчику, чтобы добраться до верхней ступеньки. $N$ - количество ступенек, $\le 100$, стоимость каждой ступеньки меньше 100. (контест 22, задача E)\newline
14. Кузнечик прыгает по столбикам, расположенным на одной линии на равных расстояниях друг от друга. Столбики имеют порядковые номера от $1$ до $N$. В начале Кузнечик сидит на столбике с номером $1$. Он может прыгнуть вперед на расстояние от $1$ до $K$ столбиков, считая от текущего. Требуется найти количество способов, которыми Кузнечик может добраться до столбика с номером $N$ . Учитывайте, что Кузнечик не может прыгать назад. $1 \le N, K \le 32$. (контест 22, задача F)\newline
\end{document}
